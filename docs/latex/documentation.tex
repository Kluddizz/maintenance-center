\documentclass[12pt]{scrreprt}
\title{Dokumentation}
\subtitle{Bilfinger Wartungscenter}
\author{Florian Hansen}

\usepackage[colorlinks,urlcolor=blue, linkcolor=black]{hyperref}
\usepackage[inline]{enumitem}
\usepackage{setspace}
\usepackage[utf8]{inputenc}
\usepackage[ngerman]{babel}

\setlist{nosep,before=\vspace{\baselineskip},after=\vspace{\baselineskip}}
\setlist[enumerate]{label*=\arabic*.}
\newcommand{\itemh}[1]{{\bfseries\large\item #1}}

\onehalfspacing

\begin{document}
  \maketitle

  \chapter{Einleitung}
  Das \textit{Bilfinger Wartungscenter} ist eine Web-Anwendung, die eine Verwaltung von Wartungsarbeiten ermöglicht.
  Hierfür stehen dem Benutzer verschiedene Funktionen zur Verfügung, um Wartungen einzusehen, zu bearbeiten und
  Berichte erstellen zu lassen. Das Ziel ist es, dem Benutzer einen guten Überblick über anstehende und bereits
  getätigte Wartungsarbeiten zu geben und ihm damit einen Großteil der Verwaltungsarbeit abzunehmen.

  \chapter{Anforderungen}
  In diesem Kapitel sollen die Anforderungen an das zu entwickelnde System analysiert und definiert werden. Hierbei wird
  zwischen allgemeinen, funktionalen und nicht-funktionalen Anforderungen unterschieden.

  \section{Allgemeine Anforderungen}
  \begin{enumerate}
    \itemh{Software-Architektur}
    \begin{enumerate}
      \item Die Anwendung soll mehrere Microservices erhalten, die komplett unabhängig voneinander arbeiten.
      \item Die Microservices sollen mithilfe von NodeJS / Express entwickelt werden.
      \item Die Microservices sollen mithilfe von Schnittstellen (APIs) kommunizieren.
      \item Die APIs sollen im RESTful-Design implementiert werden.
      \item Die Clientanwendung soll mithilfe von JavaScript / React entwickelt werden.
      \item Die Clientanwendung greift mithilfe der definierten RESTful-Schnittstellen auf die verschiedenen
        Microservices zu.
    \end{enumerate}

    \itemh{Systemanforderungen}
    \begin{enumerate}
      \item Die Andwendung soll plattformunabhängig sein.
      \item NodeJS soll als serverseitige Sprache verwendet werden, d.h., dass dies auf dem ausführenden System
        installiert werden muss.
      \item Auf dem ausführenden System (serverseitig) soll Docker installiert werden, um Microservices zu installieren bzw. zu laden.
    \end{enumerate}
  \end{enumerate}

  \section{Funktionale Anforderungen}
  \begin{enumerate}
    \itemh{Benutzerverwaltung}
    \begin{enumerate}
      \item Benutzer sollen sich anmelden können.
      \item Benutzer sollen über eine Rolle verfügen.
      \item Mögliche Rollen sollen sein: Verwalter, Wartungskraft.
      \item Benutzer sollen einen Benutzernamen, einen Vornamen, einen Nachnamen, eine eindeutige Identifikationsnummer
        (ID) und eine E-Mail-Adresse besitzen.
      \item Benutzer sollen nur manuell registriert werden können.
      \item Benutzer sollen ihre Daten, mit Ausnahme der ID, anpassen können.
    \end{enumerate}

    \itemh{Microservice: Authentifizierung}
    \begin{enumerate}
      \item Es soll möglich sein, sich mithilfe seiner Benutzerkennung und seines Passworts anzumelden.
      \item Nach einer erfolgreichen Anmeldung soll es möglich sein, sich mithilfe eines Tokens zu authentifizieren.
      \item Die Tokens sollen als Json Web Token (JWT)\footnote{\url{https://jwt.io/}} implementiert werden.
      \item Ein Token soll nur für eine bestimmte Zeit verwendet werden können.
      \item Nach Ablauf der Lebenszeit eines Tokens, muss sich der Benutzer erneut im System anmelden und einen neuen
        anfordern, um sich authentifizieren zu können.
    \end{enumerate}

    \itemh{Microservice: Wartungen}
    \begin{enumerate}
      \item Dieser Dienst soll nur authentifizierten Benutzern zur Verfügung stehen.
      \item Benutzer sollen Wartungen abrufen können. Dabei werden ihnen nur diejenigen zur Verfügung gestellt, die
        ihnen zugeteilt wurden.
      \item Nur Benutzer der Rolle \textit{Verwalter} sollen neue Wartungen erstellen können.
      \item Es soll als Verwalter möglich sein, Wartungen zu bearbeiten und zu verändern.
      \item Es soll als Verwalter möglich sein, Wartungen Wartungskräften zuzuornden.
      \item Es soll als Verwalter möglich sein, Kunden anzulegen.
      \item Es soll als Verwalter möglich sein, Kunden zu bearbeiten.
      \item Es soll als Verwalter möglich sein, Kunden zu löschen.
      \item Kunden sollen von Verwaltern Wartungen zugewiesen werden können.
      \item Ein Kunde besitzt einen Namen, eine Ansprechsperson (Vor- und Nachname, Telefonnummer, E-Mail, ...) und eine
        Anschrift (Ort, PLZ, Straße)
    \end{enumerate}

    \itemh{Benutzeroberfläche: Anmeldung}
    \begin{enumerate}
      \item Wenn der Benutzer nicht angemeldet ist, soll immer die Anmeldung aufgerufen werden.
      \item Der Benutzer soll seinen Benutzernamen und Kennwort eingeben können.
      \item Mithilfe eines Steuerelements soll der Anmeldeversuch durchgeführt werden können.
      \item Beim Scheitern des Anmeldeversuchs soll dementsprechendes Feedback gegeben werden.
      \item Beim erfolgreichen Einloggen soll der angemeldete Benutzer solange angemeldet bleiben, bis er sich wieder
        abmeldet.
    \end{enumerate}

    \itemh{Benutzeroberfläche: Dashboard}
    \begin{enumerate}
      \item Nur angemeldete Benutzer sollen diesen Teil der Anwendung sehen können.
      \item Das Dashboard soll dem Benutzer die Anzahl der anstehenden Wartungen in dem aktuellen Jahr anzeigen.
      \item Das Dashboard soll dem Benutzer die Anzahl der erledigten Wartungen in dem aktuellen Jahr anzeigen.
      \item Das Dashboard soll dem Benutzer die nächste anstehende Wartung anzeigen.
      \item Der Benutzer soll mithilfe eines Steuerelements schnell auf die Übersichtsseite der dementsprechenden
        Wartung gelangen können.
      \item Es soll eine Tabelle erstellt werden, die dem Benutzer die nächsten fünf anstehenden Wartungen anzeigt.
        Dabei soll der Kunde, der Ort, die Anlage und der Zeitpunkt des Termins angezeigt werden.
      \item Durch einen Klick auf einen Tabelleneintrag soll der Benutzer zur dementsprechenden Übersichtsseite der
        Wartung gelangen können.
    \end{enumerate}

    \itemh{Benutzeroberfläche: Wartungen}
    \begin{enumerate}
      \item Nur angemeldete Benutzer sollen diesen Teil der Anwendung sehen können.
      \item Eine Tabelle soll dem Benutzer alle Wartungen anzeigen, die ihm zugeordnet wurden.
      \item Der Benutzer soll die Einträge der Tabelle filtern können, z.B., um bereits abgeschlossene Arbeiten
        auszublenden.
      \item Der Benutzer soll eine Wartung als abgeschlossen markieren können.
      \item Benutzer mit der Rolle \textit{Verwalter} sollen neue Wartungen erstellen und anderen Benutzern zuweisen
        können.
      \item Wartungen sollen lediglich Kundendaten mit einem (wiederkehrenden) Zeitpunkt verbinden, um
        Wartungsarbeiten zu definieren.
      \item Beim Erstellen von Wartungen soll die Wartungsperiode und das Ende des Wartungsvertrages angegeben werden können.
      \item Verwalter sollen Wartungen löschen können.
      \item Verwalter sollen Wartungen bearbeiten können.
    \end{enumerate}

    \itemh{Benutzeroberfläche: Kunden}
    \begin{enumerate}
      \item Nur angemeldete Verwalter sollen diesen Teil der Anwendung sehen können.
      \item Verwalter sollen neue Kunden hinzufügen können.
      \item Kunden besitzen einen Namen, eine Anschrift und eine Ansprechsperson inkl. Kontaktinformationen.
      \item Verwalter sollen Kunden löschen können.
      \item Verwalter sollen Kundendaten bearbeiten können.
    \end{enumerate}
  \end{enumerate}

  \section{Nicht-funktionale Anforderungen}
  \begin{enumerate}
    \itemh{Sicherheit}
    \begin{enumerate}
      \item Server- und Client-Anwendungen sollen die Integrität, Vertraulichkeit und
        Verfügbarkeit der Informationen gewährleisten.
    \end{enumerate}

    \itemh{Performance}
    \begin{enumerate}
      \item Die Anwendung soll mit ausreichender Performance laufen.
      \item Algorithmen sind so zu wählen, dass diese eine möglichst geringe Komplexität besitzen.
    \end{enumerate}

    \itemh{Usability}
    \begin{enumerate}
      \item Die Anwendung soll mithilfe des Design-Systems \textit{Material
        Design}\footnote{\url{https://material.io/design/}}
        umgesetzt werden.
    \end{enumerate}
  \end{enumerate}
\end{document}
